\section{理论及算法}

\subsection{Levi分解}

考虑任何一个有限维李代数 $\mathfrak{g}$ 总能分解成可解李代数 $\rad(\mathfrak{g})$ 和半单李代数 $\mathfrak{s}=$的半直积。


\subsection{可解李代数刻画}

如果记 
\begin{equation*}
    \mathfrak{g}^1 = \mathfrak{g}^{(1)} = \mathfrak{g}, \quad
    \mathfrak{g}^{n+1} := [\mathfrak{g}\mathfrak{g}^{n}],\quad
    \mathfrak{g}^{(n+1)} := [\mathfrak{g}^{(n)}\mathfrak{g}^{(n)}],\quad n \in \N
\end{equation*}
那么可解李代数是那些有限导代数为零($\exists n \in \N, \mathfrak{g}^{(n)}$) 的李代数。从矩阵李代数来看
这些李代数在有限维时都可以表示到上三角矩阵的子代数。原因本质上是Lie定理

\begin{theorem}[Lie定理]
    代数闭域 $F$ 上有限维可解李代数 $\mathfrak{g}$ 的不可约表示都是一维的。特别地不可约有限可解李代数是一维的。
\end{theorem}

\begin{proof}
    不妨归纳,作为线性空间假设 $\dim \mathfrak{g} = n,n> 1$。
\end{proof}

\subsection{幂零李代数刻画}

幂零李代数可以用伴随映射来刻画,这是一个不平凡的定理称为Engel定理

\begin{theorem}[Engel定理]
    有限维李代数 $\mathfrak{g}$ 幂零当且仅当对所有 $g \in \mathfrak{g}$ 都有 $\ad_{g}$ 幂零
\end{theorem}


\subsection{Borchred代数}

考虑更加数学物理化的代数结构。例如

\heiti{Possion $n$-代数}
