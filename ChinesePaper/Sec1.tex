\section{问题背景阐述}

\subsection{回顾极坐标下黎曼度量}

简扼讲黎曼流形 $(M,g)$ 是一个光滑流形上带有对称的度量 $g \in \mathrm{Sym}_2(TM) \subset T^*M \otimes T^*M $ 。现在考虑局部坐标 $(x^i)_i$
和 $(y^i)_i$ 以及其坐标变换 $F:(x^i)_i \to (y^i)_i$ 的Fr\'echet导数 $D_F = (\frac{\partial y^i}{\partial x^j})_ij$。在两个坐标下度量有不同形式
\begin{equation*}
    g  = g_{ij} dx^i dx^j = \tilde{g}_{kl}dy^k dy^l = g_{ij}\frac{\partial x^i}{\partial y^k}\frac{\partial x^j}{\partial y^l}d y^k dy^l
\end{equation*}
沿袭Einstein求和约定(相同指标代表求和)利用唯一性可知 $\tilde{g}_{kl} = g_{ij}\frac{\partial x^i}{\partial y^k}\frac{\partial x^j}{\partial y^l}$
记 $G = (g_{ij})_{ij}$ 为度量矩阵而 $G^{-1} = (g^{ij})_{ij}$ 为其逆可知 $\tilde{G} = (\tilde{g}_{ij})_{ij} = D_{F^{-1}}^{t}G D_{F^{-1}}$。
再次利用Fr\'echet导数和切映射的联系
\begin{equation*}
    D_F D_{F^{-1}} = I_n,\quad n = \dim M, \quad 
\end{equation*} 
从而体积元 $dV_g = \sqrt{|det G_x|} dx^1\wedge\cdots\wedge d x^n $ 具有协变性(在 $F$ 保定向的情况下此时 $\det D_F >0$)
\begin{align*}
    &dV_g = \sqrt{|det G|} dx^1\wedge\cdots\wedge d x^n \tag*{(V1)}\\
    &d\widetilde{V_g} = \sqrt{|det \tilde{G}|} dy^1\wedge\cdots\wedge d y^n\\
    &d\widetilde{V_g}  =  \sqrt{|det G|} |\det D_F|^{-1} |\det D_F| dx^1\wedge\cdots\wedge d x^n  = \sqrt{|det G|} dx^1\wedge\cdots\wedge d x^n = dV_g
\end{align*}
协变性表示公式形式不依赖于坐标选择,从而 $dV_g$ 是一个内蕴的几何量只和度量 $g$ 有关。
对于流形上的函数 $f \in C^\infty(M)$ 定义其梯度是内蕴量
\begin{equation*}
    \langle \nabla f, X \rangle_g = X(f),\quad forall X \in X(M)
\end{equation*}
代入坐标即 $g_{ij}(\nabla f)^i X^j = \sum_i X^i\partial_i f$ 有 $(\nabla f)^i = \partial^i f = g^{ij}\partial_j f$。这个梯度公式是协变的
一方面可以从其定义的内蕴性得到另一方面可以直接计算

下面为了求和规则记 $\partial_i = \frac{\partial}{\partial x^i},\tilde{\partial}_i = \frac{\partial}{\partial y^i}$ 或者用合适下标表示
偏导变量类如 $\partial_{x,i}$ 等
\begin{align*}
    \tilde{g}^{ij}\tilde{\partial}_j f \tilde{\partial}_i & = \partial_k y_i g^{kl} \partial_l y_j \partial_m f \tilde{\partial}_{j}x^m \tilde{\partial}_i x^n \partial_n \\
                & = g^{kl}\partial_m f \partial_n  (\partial_k y_i \tilde{\partial}_i x^n ) (\partial_l y_j \tilde{\partial}_{j}x^m )\\
                & =  \delta_{kn} \delta_{lm} g^{kl}\partial_m f \partial_n = g^{ij}\partial_j f \partial_i 
\end{align*}

$\nabla f = \mathrm{grad} f $ 具有协变公式 $\nabla f = g^{ij}\partial_j f \partial_i $。次考虑切向量场 $X(M)$
的散度 $\mathrm{div} X$ 其可以内蕴地定义为
\begin{equation*}
    d (X \mathrel{\lrcorner} dV_g) = (\nabla X) dV_g
\end{equation*}
利用李导数的Cartan公式和 $d (dV_g) = 0$ 也可得到 $L_X (dV_g) = d (\iota_X (dV_g)) = \iota_X(d(dV_g)) =  d (X \mathrel{\lrcorner} dV_g)  = (\nabla X) dV_g$

这里 $X \lrcorner dV_g = \iota_X (dV_g)$ 是切向量场的缩并运算。

简记 $|g| = |\det G|$ 表示黎曼测度的体积膨胀性质。显式计算出
\begin{align*}
    \nabla \cdot X = \mathrm{div} X & = \star (\frac{1}{\sqrt{|g|}} d(\sum_i \sqrt{|g|}X^i(-1)^{i-1}dx^1\wedge \cdots \wedge \widehat{dx^i} \wedge \cdots \wedge dx^n))\\
    & = \star (\frac{1}{\sqrt{|g|}} \partial_i (\sqrt{|g|}X^i) dV_g)\\
    & = \frac{1}{\sqrt{|g|}} \partial_i (\sqrt{|g|}X^i)
\end{align*}
可以直接计算上述公式的内蕴性质,只需要利用矩阵微积分中的性质
\begin{align*}
    &\frac{d}{dt}\det A(t) = \Tr(A^{adj}(t)\frac{dA(t)}{dt}) = \det(A) \Tr (A^{-1}(t)\frac{dA(t)}{dt})\\
    &\frac{d}{dt} \ln |\det A(t)| \frac{1}{\det A(t)} \frac{d}{dt}\det A(t) = \Tr (A^{-1}(t)\frac{dA(t)}{dt}) \tag*{(A1)}
\end{align*}
上述关系称为Jacobi Formula。本质来自于矩阵群 $\mathrm{M}_n(\C)$ 的切向量满足 $\nabla_{T}\det (I_n)= \Tr T$ 从而计算记 $H = F^{-1}$
\begin{align*}
    \nabla  X 
    & = \frac{1}{\sqrt{|\tilde{g}|}} \tilde{\partial}_k (\sqrt{|\tilde{g}|}Y^k) = \frac{1}{|\det D_H|} \frac{1}{\sqrt{|g|}} \tilde{\partial}_k x^i \partial_i (\sqrt{|g|}|\det D_H|\partial_j y^k X^j)\\
    & = \frac{1}{\sqrt{|g|}} \partial_i (\sqrt{|g|}X^j) \partial_j y^k \tilde{\partial}_k x^i + X^i (\partial_i \ln |\det D_H| + \tilde{\partial}_k x^j \partial_{ij} y^k)\\
    & = \frac{1}{\sqrt{|g|}} \partial_i (\sqrt{|g|}X^j) \delta_{ij} + X^i (\Tr(D_F (\partial_i D_H) )+ \tilde{\partial}_k x^j \partial_{ij} y^k) \,\quad  \text{利用矩阵微积分A1}\\
    & = \frac{1}{\sqrt{|g|}} \partial_i (\sqrt{|g|}X^j) + X^i(\partial_k y^j \partial_i (\tilde{\partial}_j x^k) + \tilde{\partial}_j x^k \partial_i (\partial_k y^j)) \quad \text{展开} \\
    & = \frac{1}{\sqrt{|g|}} \partial_i (\sqrt{|g|}X^j) + X^i \partial_i (\partial_k y^j \tilde{\partial}_j x^k),\quad \text{莱布尼兹法则}\\
    & =  \frac{1}{\sqrt{|g|}} \partial_i (\sqrt{|g|}X^j) + X^i \partial_i(\dim M) =  \frac{1}{\sqrt{|g|}} \partial_i (\sqrt{|g|}X^j) \tag*{(A2)}
\end{align*}
从而说明了散度公式的协变性。

从梯度和散度出发,对于函数 $f \in C^\infty (M)$ 定义Laplace-Beltrami算子是两者复合
\begin{equation*}
    \Delta_g f  := \mathrm{div} \circ \mathrm{grad}f
\end{equation*}
写成协变的形式就是
\begin{equation*}
    \Delta_g f := \frac{1}{\sqrt{|g|}} \partial_i (\sqrt{|g|}g^{ij}\partial_j f )
\end{equation*}
根据梯度和散点的协变性可知 $\Delta_g f$ 不依赖于坐标选择,事实上算子
\begin{equation*}
    \Delta_g := \frac{1}{\sqrt{|g|}} \partial_i (\sqrt{|g|}g^{ij}\partial_j ) \quad \tag*{(V3)}
\end{equation*}
也是协变的,只需类似验证散点的协变性计算。

利用 $\tilde{g}^{ij} \tilde{\partial}_j = \partial_j y^i( g^{jk} {\partial}_k)$ 如果取 
$Y^i = \tilde{g}^{ij} \tilde{\partial}_j $ 和 $X^i = g^{ij}{\partial}_j $ 则自然满足协变性 $Y^i = \partial_j y^i X^j$。这
正是切向量的坐标变换方式,代入在(A2) 中的计算可知
\begin{align*}
    \widetilde{\Delta_g} &= \frac{1}{\sqrt{\tilde{g}|}} \tilde{\partial}_i (\sqrt{|\tilde{g}|}\tilde{g}^{ij}\tilde{\partial}_j ) = \frac{1}{\sqrt{|\tilde{g}|}} \tilde{\partial}_k (\sqrt{|\tilde{g}|}Y^k)\\
    & = \frac{1}{\sqrt{|g|}} \partial_i (\sqrt{|g|}X^i ) =  \frac{1}{\sqrt{|g|}} \partial_i (\sqrt{|g|}g^{ij}\partial_j )
\end{align*}
就得到了Laplace-Beltrami算子的协变性。

对于典范的黎曼模型 $(M,g)$ 其度规在标准坐标下是 $g = g_ij dx^i dx^j$ 换算成极坐标$(r,\theta)$下可以写成
\begin{align*}
    &g_M := dr^2 + \psi(r)^2 g_{\mathbb{S}^{n-1}},\quad\\
    &g_{\mathbb{S}^{n-1}} = \langle \nabla_{\theta}\frac{x}{r},\nabla_{\theta}\frac{x}{r}\rangle
    = (\sum_k\partial_{\theta_i}(\frac{x_k}{r})\partial_{\theta_j}(\frac{x_k}{r}))d\theta^i d\theta^j
    = \gamma_{ij}d\theta^i d\theta^j \\
    &\text{是球面的标准欧式度量}
\end{align*}
那么可以计算出 $(M,g)$ 的体积形式,其不依赖于坐标,选择极坐标 $(r,\theta)$ 其度量矩阵形如
\begin{equation*}
    G = 
    \left(
        \begin{matrix}
            1   & \\
                &\psi^{2}(r)G_{\mathbb{S}^{n-1}}\\
        \end{matrix}
    \right),\quad G^{-1} = 
    \left(
        \begin{matrix}
            1   & \\
                &\psi^{-2}(r)G_{\mathbb{S}^{n-1}}^{-1}\\
        \end{matrix}
    \right)
\end{equation*}
从而利用上述的内蕴公式(V1)计算出此时
\begin{equation*}
    d \nu = dV_g = \sqrt{|\det G|}dr \wedge d\theta = \psi^{n-1}(r)dr \wedge (\sqrt{|\det G_{\mathbb{S}^{n-1}}}|d\theta) = \psi^{n-1}(r)dr \wedge d \sigma_{\mathbb{S}^{n-1}}
\end{equation*}
其中 $d \sigma_{\mathbb{S}^{n-1}} = \sqrt{|\det g_{\mathbb{S}^{n-1}|}}d\theta$ 是标准球面的黎曼测度。另外利用此矩阵结构计算极坐标下的
Laplace算子利用协变公式(V3)可以得到
\begin{equation*}
    \Delta_g := \frac{1}{\sqrt{|g|}} \partial_i (\sqrt{|g|}g^{ij}\partial_j )
    =     \frac{\partial_r (\psi^{n-1} |g_{\mathbb{S}^{n-1}}| \partial_r)}{\psi^{n-1} |g_{\mathbb{S}^{n-1}}|}  
    + \frac{\partial_{\theta_i} ( \psi^{n-1} \psi^{-2} g_{\mathbb{S}^{n-1}}^{ij} \partial_{\theta_j})}{\psi^{n-1} |g_{\mathbb{S}^{n-1}}|}
    = \frac{1}{\psi^{n-1}}\partial_r (\psi^{n-1}\partial_r) + \frac{1}{\psi^2}\Delta_{\mathbb{S}^{n-1}}
\end{equation*}
也就是
\begin{equation*}
    \Delta_g := \frac{\partial^2}{\partial r^2} + \frac{\partial }{\partial r}(\ln \psi^{n-1}(r)) \frac{\partial }{\partial r} + \frac{1}{\psi^2(r)}\Delta_{\mathbb{S}^{n-1}}
\end{equation*}


\subsection{李导数和流形上的李代数}


\subsection{流形积分和Hodge对偶}
