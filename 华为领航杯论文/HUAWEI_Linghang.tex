%%%  本笔记完全参考morelulll.cls
\documentclass[UTF8, twoside, fontset=none]{ctexart}

%%% 宏
\usepackage[headheight=127mm, left=2cm, right=2cm, bottom=2cm]{geometry}    % headheight过小不被允许故设置了一个下限
\usepackage{subcaption}
\usepackage{graphicx}
% \usepackage{cite}
\usepackage{setspace}
\usepackage{multicol}            
\usepackage{multirow} 
% \usepackage{url}
\usepackage{makecell}
% \usepackage{enumitem}
\usepackage{diagbox}
\usepackage{stfloats}
\usepackage{threeparttable}




%%% 字体设置 %%%
\usepackage[T1]{fontenc}
\usepackage{textcomp}  % for \textquotesingle macro
% 英文等宽字体
% \setmonofont{Fantasque Sans Mono}
\usepackage{xeCJK}
\setCJKmainfont{simsun.ttc}
% \setCJKsansfont{}
\setCJKmonofont{simfang.ttf}
\setCJKfamilyfont{zhhei}{simhei.ttf}[BoldFont=simhei.ttf, ItalicFont=fantasquesansmono-bolditalic.otf]
\setCJKfamilyfont{zhyou}{FZY1K.TTF}[BoldFont=FZY1K.TTF, ItalicFont=FZY3K.TTF]
\setCJKfamilyfont{方正楷体}{FZKTJW.TTF}
\newcommand{\heiti}[1]{{\CJKfamily{zhhei}{#1}}}
\newcommand{\youyuan}[1]{{\CJKfamily{zhyou}{#1}}}



%%% 链接设置
% 定义颜色
\usepackage{xcolor}
\definecolor{DarkRed}{RGB}{139,0,0}
\definecolor{DimGrey}{RGB}{105 105 105}
\definecolor{ShallowGray}{RGB}{220, 220, 220}
\definecolor{CommentGreen}{RGB}{0, 255, 0}
\definecolor{KeyMauve}{RGB}{204, 153, 255}
\definecolor{thmcolor}{RGB}{241, 241, 255}
\definecolor{defcolor}{RGB}{100, 149, 237}
% 超链接设置
\usepackage[
    colorlinks=true,
    linkcolor=DarkRed,
    linkbordercolor=DarkRed,
    anchorcolor=DimGrey,
    citecolor=DarkRed,
    plainpages=false,
    pdfencoding=unicode,
    hypertexnames=true,
]{hyperref}
% 定理环境
\usepackage[fleqn]{amsmath}     %   公式左对齐
\usepackage{amssymb,amsthm,amsfonts}
\usepackage{mathtools}      % 数学符号美化
\usepackage{tikz-cd}
\usepackage{verbatim}       % 注释
\allowdisplaybreaks[4]      % 公式跨页
\renewcommand\theequation{\thetheorem\alph{equation}}
\newtheorem{theorem}{定理}[section]
\newtheorem{proposition}[theorem]{\heiti{命题}}
\newtheorem{corollary}{\youyuan{推论}}[theorem]
\newtheorem{lemma}{\youyuan{引理}}[theorem]
\newtheorem{definition}{$\blacksquare$ \,定义}[section]



%%% 页眉页脚设置
\usepackage{fancyhdr}
\pagestyle{fancy}
\fancyhead[C]{\textcolor{black}{\leftmark}}
\fancyhead[RO,LE]{\textcolor{DimGrey}{\thepage}}
\fancyhead[RE,LO]{}
\fancyfoot{}
\renewcommand{\headrule}{\color{DimGrey}\hrule width\textwidth}

%   文本提示
\usepackage{tcolorbox}
\tcbuselibrary{breakable,skins}
\tcbsetforeverylayer{enhanced}
\newtcolorbox[auto counter,number within=section]{notebox}[1]{
    skin=empty,
    top = 0pt,
    bottom = 0pt,
    toprule = 0pt,
    bottomrule = 0pt,
    leftrule = 0pt,
    rightrule = 0pt,
    borderline west={2pt}{0pt}{#1},
    breakable, %允许跨页
}

% 注
\newenvironment{note}
{\begin{notebox}{orange}
      \textcolor{orange}{ \CJKfamily{zhhei}{注:}}
}
{\end{notebox}}

% 例
\newenvironment{case}
{\begin{notebox}{blue}
      \textcolor{blue}{ \CJKfamily{zhhei}{例:}}
}
{\end{notebox}}

% 论断
\newtcolorbox{claim}{colframe = blue!75!black}

% 引用结论
\renewenvironment{quote}
{\begin{notebox}{gray}
      \textcolor{gray}{ \CJKfamily{zhhei}{\textbullet}}
}
{\end{notebox}}



%%% 参考文献配置
\usepackage[backend=biber,backref=true,bibstyle=draft,citestyle=ext-authoryear]{biblatex}
\DeclareOuterCiteDelims{cite}{\bibopenbracket}{\bibclosebracket}
\AtEveryCite{\color{blue}}      %   蓝色引用文字
\addbibresource{HUAWEI_Linghang.bib}



%%% 命令简化
\renewcommand{\O}{\mathcal{O}}
\renewcommand{\contentsname}{\heiti{论文目录}}  %   重新设置目录标题
\newcommand{\C}{\mathbb{C}} %   复数
\newcommand{\N}{\mathbb{N}}     %   自然数
\newcommand{\R}{\mathbb{R}}     %   实数
\newcommand{\Z}{\mathbb{Z}}     %   整数
\def\x{{\mathbf x}}
\def\L{{\cal L}}
\newcommand{\tabincell}[2]{\begin{tabular}{@{}#1@{}}#2\end{tabular}}    %   表


%%% 文件树呈现
\usepackage{forest}



%%% 代码解释
\usepackage{listings}
\lstset{ %
  backgroundcolor=\color{ShallowGray},      % choose the background color
  basicstyle=\footnotesize,                 % size of fonts used for the code
  breaklines=true,                          % automatic line breaking only at whitespace
  captionpos=b,                             % sets the caption-position to bottom
  commentstyle=\color{CommentGreen},        % comment style
%   escapeinside={\%*}{*)},                 % if you want to add LaTeX within your code
  keywordstyle=\color{blue},                % keyword style
  stringstyle=\color{KeyMauve},             % string literal style
  frame=single,
  numbers=left,
}



%%% 扉页设置
%   预定义命令
\NewDocumentCommand{\thesistitle}{ o m }{
    \IfValueTF{#1}{\def\shorttitle{#1}}{\def\shorttitle{#2}}
    \def\@title{#2}
    \def\ttitle{#2}
    }
% \DeclareDocumentCommand{\author}{m}{\newcommand{\authorname}{#1}\renewcommand{\@author}{#1}}
\NewDocumentCommand{\supervisor}{m}{\newcommand{\supname}{#1}}
\NewDocumentCommand{\examiner}{m}{\newcommand{\examname}{#1}}
\NewDocumentCommand{\degree}{m}{\newcommand{\degreename}{#1}}
\NewDocumentCommand{\addresses}{m}{\newcommand{\addressname}{#1}}
\NewDocumentCommand{\university}{m}{\newcommand{\univname}{#1}}
\NewDocumentCommand{\department}{m}{\newcommand{\deptname}{#1}}
\NewDocumentCommand{\group}{m}{\newcommand{\groupname}{#1}}
\NewDocumentCommand{\faculty}{m}{\newcommand{\facname}{#1}}
\NewDocumentCommand{\subject}{m}{\newcommand{\subjectname}{#1}}
\NewDocumentCommand{\keywords}{m}{\newcommand{\keywordnames}{#1}}
\newcommand{\HRule}{\rule{.9\linewidth}{.6pt}} % New command to make the lines in the title page
%	论文信息
\thesistitle{Low Dissipation and Adaptive FIR Filter Algorithms} % Your thesis title, this is used in the title and abstract, print it elsewhere with \ttitle
\supervisor{\textsc{REK3000}} % Your supervisor's name, this is used in the title page, print it elsewhere with \supname
\examiner{} % Your examiner's name, this is not currently used anywhere in the template, print it elsewhere with \examname
\degree{HUAWEI Applied Mathematics Contest} % Your degree name, this is used in the title page and abstract, print it elsewhere with \degreename
\addresses{Shanghai,China} % Your address, this is not currently used anywhere in the template, print it elsewhere with \addressname
\subject{Hardware Algorithms} % Your subject area, this is not currently used anywhere in the template, print it elsewhere with \subjectname
\keywords{} % Keywords for your thesis, this is not currently used anywhere in the template, print it elsewhere with \keywordnames
\university{\href{https://yotally.github.io/}{HUAWEI APPLIED MATHEMATICAL CONTEST}} % Your university's name and URL, this is used in the title page and abstract, print it elsewhere with \univname
\department{\href{null}{Chai Hao in SMCS,Fudan University,\\
Li Minghao in EIE,Nanjiing University,\\
Wang Xiyuan in SMS,Fudan University}} % Your department's name and URL, this is used in the title page and abstract, print it elsewhere with \deptname
\group{\href{https://yotally.github.io/}{Fudan University, Nanjing University}} % Your research group's name and URL, this is used in the title page, print it elsewhere with \groupname
\faculty{\href{https://github.com/YOTALTEAM}{REK3000}} % Your faculty's name and URL, this is used in the title page and abstract, print it elsewhere with \facname
%   文档起始设置超链接格式
\AtBeginDocument{
	\hypersetup{pdftitle=\ttitle} % Set the PDF's title to your title
    % \hypersetup{pdfauthor=\authorname} % Set the PDF's author to your name
	\hypersetup{pdfkeywords=\keywordnames} % Set the PDF's keywords to your keywords
	\hypersetup{hypertexnames=true} % Make bib index back reference a right position
}

\title{ \heiti{华为领航杯应用数学大赛论文}\\
低功耗自适应有限脉冲滤波器算法}

\author{} % Your name, this is used in the title page and abstract, print it elsewhere with \authorname

\date{2023.8}

\begin{document}

% \frontmatter % Use roman page numbering style (i, ii, iii, iv...) for the pre-content pages

\pagestyle{plain} % Default to the plain heading style until the thesis style is called for the body content

%----------------------------------------------------------------------------------------
%	TITLE PAGE
%----------------------------------------------------------------------------------------

\begin{titlepage}
    \begin{center}
    
    \vspace*{.06\textheight}
    {\scshape\LARGE \univname\par}\vspace{1.5cm} % University name
    \textsc{\Large Contest Thesis}\\[0.5cm] % Thesis type
    
    \HRule \\[0.4cm] % Horizontal line
    {\LARGE \bfseries \ttitle\par}\vspace{0.4cm} % Thesis title
    \HRule \\[1.5cm] % Horizontal line
     
    \begin{minipage}[t]{0.4\textwidth}
    \begin{flushleft} \large
    \emph{Author:}\\
    \href{https://github.com/YOTALTEAM}{Chai Hao, \\
                                    Li Minghao, \\
                                    Wang Xiyuan} % Author name (remove the \href bracket to remove the link)
    \end{flushleft}
    \end{minipage}
    \begin{minipage}[t]{0.4\textwidth}
    \begin{flushright} \large
    \emph{Team:} \\
    \href{https://github.com/YOTALTEAM}{\supname} % Supervisor name - remove the \href bracket to remove the link  
    \end{flushright}
    \end{minipage}\\[3cm]
     
    \vfill
    
    \large \textit{A thesis submitted in fulfillment of the requirements\\ for the \degreename Applied Mathematics Contest}\\[0.3cm] % University requirement text
    \textit{Question Number 7 in August 2023}\\[0.4cm]
    \groupname \\ \deptname \\[2cm] % Research group name and department name
     
    \vfill
    \end{center}
\end{titlepage}

\maketitle

\begin{abstract}
    本文主要实现一种低功耗(Low Dissipation)自适应(Adaptive)的有限长单位脉冲响应滤波器(Finite Impulse 
    Response Filter)算法。主要参考文献是 \cite{Winograd1968,Gholami2021,Jiang2020,Cormen2022,Nussbaumer1982}。

    低功耗算法的主要思想是两类:一是等价改变滤波器架构最小化乘法器个数;二是降低滤波器的乘法器的运算复杂度。其中
    架构的不同实现包含 Cooley-Tukey 算法、内积加速算法\cite{Winograd1968}、Karatsuba 快乘\cite{Karatsuba1995}
    (主要是通过减少乘法而加速离散傅立叶变换,通论可见快速\cite[Chap1,2]{Nussbaumer1982});加速乘法器本身的算法诸如
    tooth加速乘法\cite{Li2010}、近似乘法和量化感知优化\cite{Gholami2021}(中文写成的数值计算神书
    \cite{Li2010}有详尽说明)。

    自适应算法采用机器学习预测算法,主要保证快速校正和模型无损(题目的诉求三)。通过预测信号量进行自适应滤波能够输出指定的
    或期望的样式信号,以达到降噪、降低功耗、快速响应等实时场景的不同要求。
    
    本文是第一届华为领航杯应用数学大赛参数论文。本文之格式内容均依照该竞赛要求完成。
\end{abstract}

\newpage 

%   目录标题颜色
{\color{DarkRed}\tableofcontents}

\newpage

\listoffigures

\newpage

\clearpage

% \mainmatter % Begin numeric (1,2,3...) page numbering

\section{\heiti{问题背景阐述}}

\subsection{\heiti{有限脉脉冲响应滤波器(Finite Impulse Response Filter, FIR Filter)}}

工程上进行
FIR 滤波器(FIR Filter,以下简称FIR或FIR滤波器)是非递归型滤波器的简称,全称是有限长单位脉冲响应滤波器
(Finite Response Filter)。

其中带有常数系数的
如下图




\subsection{\heiti{离散傅立叶变换及快速实现(Discrete/Fast Fourier Transformatioin, DFT/FFT)}}

如果采用上述一般形式的离散 Fourier 变换公式在硬件上进行运算,那么完成一次完整的 DFT 需要进行
$\O (n^2)$ 次乘法和 $\O(n^2)$ 次加法。其中 $n$ 代表 DFT 变换器的阶数(或者滤波器的抽头数)

在20世纪60年代,来自 Princeton 的计算机科学家 Cooley 和数学家 Tukey 发明了一种基于快速幂算法和 Gauss 和算术性质
的快速离散 Fourier 算法。其可以通过数量级地减少 DFT 算法中的乘法次数
大大加快原有离散 Fourier 变换的运算速度。本质上这一算法的理论框架已经被德国数学家 Carl Friedrich Gauss 于 19世纪初
发现并证明,参见。

\subsection{\heiti{滤波器算法中的乘法器性能}}


\subsection{\heiti{技术诉求}}

针对华为领航杯应用数学大赛的难题``低功耗自适应 FIR 滤波器''有如下三个主要技术诉求。

\begin{enumerate}
    \item 诉求一:通过改变滤波器架构设计等办法,使得其硬件算法整体性能提升40\%。
    \item 诉求二:指定位宽下引入近似乘法,实现乘法资源降低50\%,误差或信噪比(SNR)大于 $75 \mathrm{dBc}$
            ;或者使用机器学习模型训练,使得其训练
            性能基本无损($< 0.1 \mathrm{dB}$)
    \item 诉求三:" 低位宽 " FIR 滤波器算法实现($< 6 \mathrm{bit}$),并能够实现模型稳定校正且性能基本无损($< 0.1 \mathrm{dB}$)
\end{enumerate}

在本文第二部分经典方法复现中,三个子节将会依次详细论述本项目对上述三个诉求的经典解决方式。

在第三部分``理论和算法''中本文将尽可能清晰地叙述项目中使用的关键算法理论,具体创新和优化的细节部分。

在第四部分``实验数据及软硬件介绍''和第五部分``实验结果及主要结论''中本文将展示根据前三部分的算法理论实现的工程在仿真环境
的测试数据以及指标比较。

最后关于项目的复现以及移植,完整的源代码解释和运行环境信息全面呈现在附录一节。



\section{经典方法复现}


\section{理论及算法}

\subsection{离散傅立叶变换(Discrete Fourier Transformatioin, FFT) 的快速实现}

快速傅立叶变换(通常只是至离散情形的)
以下内容参考wikipedia 。如



第一种办法:Wingoard 加速算法(Wingoard Fourier Transformation,WFT)

第二种办法:Cooley-Tukey 算法(Cooley-Tukey Algorithms)

此方式适用于 $n= 2^m$ 时,这里的实现方式主要参考了算法导论一书的 FFT 章节见 \cite[Polynomials and FFT]{Cormen2022}

另外更加高效的办法是参考 \cite{Li2010}。或者关于快速傅立叶变换的综述性质书籍\cite{Nussbaumer1982}。

第三种办法
\subsection{乘法器加速的 Booth 算法(Booth Multiplication Algorithms)}


\subsection{量化感知中的混合精度量化(Mixed Precision Quantization)}




\section{\heiti{实验数据及软硬件介绍}}


\subsection{\heiti{基于昇腾/ARM 架构的 xxx 型号芯片的仿真测试数据}}

\subsection{\heiti{基于 OpenNN/PyTorch 框架的量化感知算法测试数据}}

以下是神经网络的测试算法。

\subsection{\heiti{实验用软硬件信息}}

\section{\heiti{实验结果及主要结论}}

\subsection{\heiti{主要测试结果}}


\subsection{\heiti{结论}}

实现性能 $40 \% $ 的提升的结论。


\section{总结及展望}


有限长单位脉冲响应滤波算法是一种广泛应用在通信工程、音视频处理、信号处理等工程方向的硬件算法。其作为信号处理技术中
的核心主流算法,已经被多系列主流基带芯片通信芯片所集成。如何高效执行有限长单位脉冲响应滤波算法,无论是从算法理论上
还是硬件兼容软件集成上都是学界和工业界的主攻难题。

量化感知是近二三十年来发展迅速的加速神经网络训练的主流技术之一。从提出至今已经有不同研究者提出了种类繁多的量化策略
其想要解决的主要问题是在一些性能限制下最小化如下的量化后数据和原始数据的偏差

\begin{equation*}
    \min \sum_{r} \| Q(r) - r\|_2,\qquad \mathrm{s.t.}\quad \mathrm{Constraints(r) = 0}
\end{equation*}

高效的量化感知算法可以大大加快神经网络训练的底层运算速度,是一个加速神经网络训练的可行方向。但是另一方面,
神经网络自身的结构优劣和误差传播算法的设计好坏也是制约其训练质量和速度的重要指标。从这一意义上说,设计更好的
神经网络结构和与之配套的误差传播/激励传播算法可能会从另一个方向优化神经网络的训练。这种优化可能达到量化感知
技术无法实现的程度。

综观工程技术发展历史,最底层的算法和技术迭代往往艰深险难,每进步一毫可能都要耗费巨大的时间人力成本,更甚者算法层面
的进步可能需要数学理论的艰难创新或是天才般的灵光乍现。本文集成了快速傅立叶变换算法、低功耗约束的高效规划算法、量化感知
加速技术等诸多经典或者前沿的工程算法,其中不妨极具影响力的`世界级'算法。经过软硬件整合实现一套可以在华为系硬件芯片上
运行的完整的自适应低功耗有限长单位脉冲响应滤波算法(Self-Adjusting Low-Dissipation Finite Impulse Response Filter)

附录中包含了可执行的C++工程的源代码实现以及详细注释;量化感知技术实现的神经网络框架(基于OpenNN/Tensorflow)源代码实现
和接口注释;以及详细的操作方式。(API Source Code;Description;Manual)

如果有后续的经费支持,我们团队认为我们可能继续维护该项目的开发,并且生产出更加高效的算法。

\section*{\heiti{附录}}

这是附录文件。这是附录文件。这是附录文件。这是附录文件。这是附录文件。
这是附录文件。这是附录文件。这是附录文件。这是附录文件。这是附录文件。这是附录文件。
这是附录文件。这是附录文件。这是附录文件。这是附录文件。这是附录文件。这是附录文件。
这是附录文件。这是附录文件。这是附录文件。这是附录文件。这是附录文件。这是附录文件。
这是附录文件。这是附录文件。这是附录文件。这是附录文件。这是附录文件。这是附录文件。
这是附录文件。这是附录文件。这是附录文件。这是附录文件。这是附录文件。这是附录文件。

\subsection{\heiti{源程序文件树}}
全工程文件树如下图所示

\begin{center}
    \begin{forest}
        for tree={
          font=\ttfamily,
          grow'=0,
          child anchor=west,
          parent anchor=south,
          anchor=west,
          calign=first,
          edge path={
            \noexpand\path [draw, \forestoption{edge}]
            (!u.south west) +(7.5pt,0) |- node[fill,inner sep=1.25pt] {} (.child anchor)\forestoption{edge label};
          },
          before typesetting nodes={
            if n=1
              {insert before={[,phantom]}}
              {}
          },
          fit=band,
          before computing xy={l=15pt},
        }
      [HUAWEI\_FIR
        [ACFIR
          [...]
        ]
        [MLPyPlugin
          [...]
        ]
        [VDHL
          [...]
        ]
        [etc]
      ]
    \end{forest}    
\end{center}

其中 ACFIR 文件夹是可运行在硬件或仿真系统上的 FIR 滤波算法源代码; MLPyPlugin 文件夹是优化滤波器性能的机器学习
 Python 代码文件;VDHL 是用于仿真模拟或硬件测试的硬件描述语言源代码。``etc''表示其他诸如测试数据,日志文件,
编译软件生成文件等附加的文件。``...''表示文件夹中存在的没有列举出的文件。

关于本论文和后续汇报的幻灯片文件将在 GitHub 团队 \href{github.com/YOTALTEAM}{地址}
上的 Contest-Piece 仓库中的``华为领航杯论文''文件夹中发布。由于自身嵌套问题此文件恕不在附录的源程序文件解释部分解释,
也于文件树上省略此文件夹。

同时此比赛的一切源代码均发布到指定邮箱。

$>>>$ 此处添加ACFIR文件树


\begin{center}
    \begin{forest}
        for tree={
          font=\ttfamily,
          grow'=0,
          child anchor=west,
          parent anchor=south,
          anchor=west,
          calign=first,
          edge path={
            \noexpand\path [draw, \forestoption{edge}]
            (!u.south west) +(7.5pt,0) |- node[fill,inner sep=1.25pt] {} (.child anchor)\forestoption{edge label};
          },
          before typesetting nodes={
            if n=1
              {insert before={[,phantom]}}
              {}
          },
          fit=band,
          before computing xy={l=15pt},
        }
      [ACFIR
        [benchmark.h]
        [benchmark.cpp]
        [...]
      ]
    \end{forest}    
\end{center}

程序解答

\paragraph{transformation.h}

这是表示引用文件。

\begin{lstlisting}[language=java]
    class HelloWorldApp {
        public static void main(String[] args) {
            System.out.println("Hello World!"); // Display the string.
            for (int i = 0; i < 100; ++i) {
                System.out.println(i);
            }
        }
    }
\end{lstlisting}
    

$>>>$ 此处添加Python机器学习框架文件库

\begin{center}
    \begin{forest}
        for tree={
          font=\ttfamily,
          grow'=0,
          child anchor=west,
          parent anchor=south,
          anchor=west,
          calign=first,
          edge path={
            \noexpand\path [draw, \forestoption{edge}]
            (!u.south west) +(7.5pt,0) |- node[fill,inner sep=1.25pt] {} (.child anchor)\forestoption{edge label};
          },
          before typesetting nodes={
            if n=1
              {insert before={[,phantom]}}
              {}
          },
          fit=band,
          before computing xy={l=15pt},
        }
      [MLPyPlugin
        [data\_config.py]
        [torch\_config.py]
        [run\_strategy.wheel]
        [...]
      ]
    \end{forest}
\end{center}

$>>>$ 其他

程序解读

\subsubsection{滤波器代码注解}

下面的 C++ 函数实现了滤波器的基准测试功能。

使用信号处理的名字空间

\paragraph{convolution.cpp}

\begin{center}
    \begin{lstlisting}[language=C++]
        #include "convolution.h"
    
        namespace dsplib{
            namespace conv{
                vfp fir_regular(const vfp& seq, const vfp& tap) {
                    int slen = seq.size();
                    int tlen = tap.size();
                    int olen = slen + tlen - 1;
                    vfp vo(olen);
                    for(int i = 0; i < olen; i++) {
                        vo[i] = 0;
                        for(int t = 0; t <= i; t++) {
                            if((t < tlen) && ((i - t) < slen))
                                vo[i] += tap[t] * seq[i - t];
                        }
                    }
                    return vo;
                };
            }
        }
    \end{lstlisting}    
\end{center}
其中等等。

\subsubsection{机器学习框架代码注解}

此时使用 torch 包对滤波器参数选择进行神经网络优化。参考

\begin{center}
    \begin{lstlisting}[language=python]
        from __future__ import division
        import gym
        import numpy as np
        import torch
        from torch.autograd import Variable
        import os
        import psutil
        import gc
        import random
        import train
        import buffer
        import matplotlib.pyplot as plt
        import pandas as pd
        from tqdm import tqdm
        import sys
    \end{lstlisting}
\end{center}

\subsubsection{可视化框架注解}

这里使用

\subsection{\heiti{使用指南}}


\subsubsection{操作系统和软件要求}


\subsubsection{执行步骤}


$>>>$ run xxx.exe

拟代码(pseudo code)


StepA,  StepB,      .....   C ....

\phantomsection
\addcontentsline{toc}{section}{\heiti{附录}}

\newpage
\phantomsection
\addcontentsline{toc}{section}{\heiti{参考文献}}
\printbibliography[title={\heiti{参考文献}}]


\end{document}
