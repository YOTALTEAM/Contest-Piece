\section*{\heiti{附录}}

这是附录文件。这是附录文件。这是附录文件。这是附录文件。这是附录文件。
这是附录文件。这是附录文件。这是附录文件。这是附录文件。这是附录文件。这是附录文件。
这是附录文件。这是附录文件。这是附录文件。这是附录文件。这是附录文件。这是附录文件。
这是附录文件。这是附录文件。这是附录文件。这是附录文件。这是附录文件。这是附录文件。
这是附录文件。这是附录文件。这是附录文件。这是附录文件。这是附录文件。这是附录文件。
这是附录文件。这是附录文件。这是附录文件。这是附录文件。这是附录文件。这是附录文件。

\subsection{\heiti{源程序文件树}}
全工程文件树如下图所示

\begin{center}
    \begin{forest}
        for tree={
          font=\ttfamily,
          grow'=0,
          child anchor=west,
          parent anchor=south,
          anchor=west,
          calign=first,
          edge path={
            \noexpand\path [draw, \forestoption{edge}]
            (!u.south west) +(7.5pt,0) |- node[fill,inner sep=1.25pt] {} (.child anchor)\forestoption{edge label};
          },
          before typesetting nodes={
            if n=1
              {insert before={[,phantom]}}
              {}
          },
          fit=band,
          before computing xy={l=15pt},
        }
      [HUAWEI\_FIR
        [ACFIR
          [...]
        ]
        [MLPyPlugin
          [...]
        ]
        [VDHL
          [...]
        ]
        [etc]
      ]
    \end{forest}    
\end{center}

其中 ACFIR 文件夹是可运行在硬件或仿真系统上的 FIR 滤波算法源代码; MLPyPlugin 文件夹是优化滤波器性能的机器学习
 Python 代码文件;VDHL 是用于仿真模拟或硬件测试的硬件描述语言源代码。``etc''表示其他诸如测试数据,日志文件,
编译软件生成文件等附加的文件。``...''表示文件夹中存在的没有列举出的文件。

关于本论文和后续汇报的幻灯片文件将在 GitHub 团队 \href{github.com/YOTALTEAM}{地址}
上的 Contest-Piece 仓库中的``华为领航杯论文''文件夹中发布。由于自身嵌套问题此文件恕不在附录的源程序文件解释部分解释,
也于文件树上省略此文件夹。

同时此比赛的一切源代码均发布到指定邮箱。

$>>>$ 此处添加ACFIR文件树


\begin{center}
    \begin{forest}
        for tree={
          font=\ttfamily,
          grow'=0,
          child anchor=west,
          parent anchor=south,
          anchor=west,
          calign=first,
          edge path={
            \noexpand\path [draw, \forestoption{edge}]
            (!u.south west) +(7.5pt,0) |- node[fill,inner sep=1.25pt] {} (.child anchor)\forestoption{edge label};
          },
          before typesetting nodes={
            if n=1
              {insert before={[,phantom]}}
              {}
          },
          fit=band,
          before computing xy={l=15pt},
        }
      [ACFIR
        [benchmark.h]
        [benchmark.cpp]
        [...]
      ]
    \end{forest}    
\end{center}

程序解答

\paragraph{transformation.h}

这是表示引用文件。

\begin{lstlisting}[language=java]
    class HelloWorldApp {
        public static void main(String[] args) {
            System.out.println("Hello World!"); // Display the string.
            for (int i = 0; i < 100; ++i) {
                System.out.println(i);
            }
        }
    }
\end{lstlisting}
    

$>>>$ 此处添加Python机器学习框架文件库

\begin{center}
    \begin{forest}
        for tree={
          font=\ttfamily,
          grow'=0,
          child anchor=west,
          parent anchor=south,
          anchor=west,
          calign=first,
          edge path={
            \noexpand\path [draw, \forestoption{edge}]
            (!u.south west) +(7.5pt,0) |- node[fill,inner sep=1.25pt] {} (.child anchor)\forestoption{edge label};
          },
          before typesetting nodes={
            if n=1
              {insert before={[,phantom]}}
              {}
          },
          fit=band,
          before computing xy={l=15pt},
        }
      [MLPyPlugin
        [data\_config.py]
        [torch\_config.py]
        [run\_strategy.wheel]
        [...]
      ]
    \end{forest}
\end{center}

$>>>$ 其他

程序解读

\subsubsection{滤波器代码注解}

下面的 C++ 函数实现了滤波器的基准测试功能。

使用信号处理的名字空间

\paragraph{convolution.cpp}

\begin{center}
    \begin{lstlisting}[language=C++]
        #include "convolution.h"
    
        namespace dsplib{
            namespace conv{
                vfp fir_regular(const vfp& seq, const vfp& tap) {
                    int slen = seq.size();
                    int tlen = tap.size();
                    int olen = slen + tlen - 1;
                    vfp vo(olen);
                    for(int i = 0; i < olen; i++) {
                        vo[i] = 0;
                        for(int t = 0; t <= i; t++) {
                            if((t < tlen) && ((i - t) < slen))
                                vo[i] += tap[t] * seq[i - t];
                        }
                    }
                    return vo;
                };
            }
        }
    \end{lstlisting}    
\end{center}
其中等等。

\subsubsection{机器学习框架代码注解}

此时使用 torch 包对滤波器参数选择进行神经网络优化。参考

\begin{center}
    \begin{lstlisting}[language=python]
        from __future__ import division
        import gym
        import numpy as np
        import torch
        from torch.autograd import Variable
        import os
        import psutil
        import gc
        import random
        import train
        import buffer
        import matplotlib.pyplot as plt
        import pandas as pd
        from tqdm import tqdm
        import sys
    \end{lstlisting}
\end{center}

\subsubsection{可视化框架注解}

这里使用

\subsection{\heiti{使用指南}}


\subsubsection{操作系统和软件要求}


\subsubsection{执行步骤}


$>>>$ run xxx.exe

拟代码(pseudo code)


StepA,  StepB,      .....   C ....

\phantomsection
\addcontentsline{toc}{section}{\heiti{附录}}