\section*{\heiti{附录}}

这是附录文件。有待于后续更新。

\subsection{\heiti{源程序文件树}}
全工程文件树如下图所示

\begin{center}
    \begin{forest}
        for tree={
          font=\ttfamily,
          grow'=0,
          child anchor=west,
          parent anchor=south,
          anchor=west,
          calign=first,
          edge path={
            \noexpand\path [draw, \forestoption{edge}]
            (!u.south west) +(7.5pt,0) |- node[fill,inner sep=1.25pt] {} (.child anchor)\forestoption{edge label};
          },
          before typesetting nodes={
            if n=1
              {insert before={[,phantom]}}
              {}
          },
          fit=band,
          before computing xy={l=15pt},
        }
      [HUAWEI\_FIR
        [ACFIR
          [...]
        ]
        [MLPyPlugin
          [...]
        ]
        [Verilog
          [...]
        ]
        [etc]
      ]
    \end{forest}    
\end{center}

其中 ACFIR 文件夹是可运行在硬件或仿真系统上的 FIR 滤波算法源代码; MLPyPlugin 文件夹是优化滤波器性能的机器学习
 Python 代码文件;VDHL 是用于仿真模拟或硬件测试的硬件描述语言源代码。``etc''表示其他诸如测试数据,日志文件,
编译软件生成文件等附加的文件。``...''表示文件夹中存在的没有列举出的文件。

关于本论文和后续汇报的幻灯片文件将在 GitHub 团队 \href{github.com/YOTALTEAM}{地址}
上的 Contest-Piece 仓库中的``华为领航杯论文''文件夹中发布。由于自身嵌套问题此文件恕不在附录的源程序文件解释部分解释,
也于文件树上省略此文件夹。

同时此比赛的一切源代码均发布到指定邮箱。

$>>>$ 此处添加ACFIR文件树

\paragraph{transformation.h} 此为 FIR 使用的变换函数的声明文件。为 C++ 头文件(Header File)。

\subsubsection{滤波器代码注解}

下面的 C++ 函数实现了滤波器的基准测试功能。重构了 DSP 数字信号处理的名字空间。

\paragraph{convolution.cpp}

\begin{center}
    \begin{lstlisting}[language=C++]
        #include "convolution.h"
    
        namespace dsplib{
            namespace conv{
                vfp fir_regular(const vfp& seq, const vfp& tap) {
                    int slen = seq.size();
                    int tlen = tap.size();
                    int olen = slen + tlen - 1;
                    vfp vo(olen);
                    for(int i = 0; i < olen; i++) {
                        vo[i] = 0;
                        for(int t = 0; t <= i; t++) {
                            if((t < tlen) && ((i - t) < slen))
                                vo[i] += tap[t] * seq[i - t];
                        }
                    }
                    return vo;
                };
            }
        }
    \end{lstlisting}    
\end{center}

\subsubsection{机器学习框架代码注解}

本项目使用 PyTorch 包对滤波器参数选择进行神经网络优化。

\subsubsection{可视化框架注解}

待更新

\subsection{\heiti{使用指南}}


\subsubsection{操作系统和软件要求}

参考源代码或 GitHub 仓库。

\subsubsection{执行步骤}

参考源代码或 GitHub 仓库。


\phantomsection
\addcontentsline{toc}{section}{\heiti{附录}}