\section{\heiti{总结及展望}}


有限长单位脉冲响应滤波算法是一种广泛应用在通信工程、音视频处理、信号处理等工程方向的硬件算法。其作为信号处理技术中
的核心主流算法,已经被多系列主流基带芯片通信芯片所集成。如何高效执行有限长单位脉冲响应滤波算法,无论是从算法理论上
还是硬件兼容软件集成上都是学界和工业界的主攻难题。

低功耗滤波器的算法实现主要依靠两部分性能的提升,一个是底层算法诸如乘法器和卷积运算器的性能提升,另一个是系统架构
升级产生的运算次数减少/内存调度优化等方式的整体性能提升。数学上抽象即在保持一些例如稳定性、准确性的条件下优化
\begin{align*}
    &\min S_{\mathrm{Integral}} + C_{\mathrm{num+res}},\qquad \mathrm{s.t.}\qquad \mathrm{Constraints(r_1,..,r_k) = 0}\\
    & \, S_{\mathrm{Integral}} \quad \text{硬件架构运算器数量等整体功耗(用功耗面积度量)}\\
    & \, C_{\mathrm{num+res}}   \quad \text{底层算法功耗(用单次运算消耗内存和时间的综合指标度量)}\\
    & \, r_1,...,r_k            \quad \text{度量稳定性准确性和其他要求的指标}
\end{align*}
自适应滤波器原理是通过预测/经验方法实现诸如 确定模型无损的方式是期望最小化如下的近似/预测/补偿误差
\begin{align*}
    &\mathrm{Find} \qquad \widehat{y} \quad \mathrm{s.t.} \, \| y - \widehat{y} \|_2 < \epsilon,\qquad\forall y \in \Omega_{D}\\
    & y,\widehat{y},\quad \text{分别表示测试信号数据和补偿信号}\\
    & \epsilon \to F_{\epsilon} = \widehat{(-)},\quad \text{表示针对不同的精度要求能够开发不同的自适应补偿算子} F_{\epsilon}
\end{align*}
量化感知是近二三十年来发展迅速的加速神经网络训练的主流技术之一。从提出至今已经有不同研究者提出了种类繁多的量化策略
其想要解决的主要问题是在一些性能限制下最小化如下的量化后数据和原始数据的偏差
\begin{equation*}
    \min \sum_{r} \| Q(r) - r \|_2,\qquad \mathrm{s.t.}\quad \mathrm{Constraints(r) = 0}
\end{equation*}
高效的量化感知算法可以大大加快神经网络训练的底层运算速度,是一个加速神经网络训练的可行方向。但是另一方面,
神经网络自身的结构优劣和误差传播算法的设计好坏也是制约其训练质量和速度的重要指标。从这一意义上说,设计更好的
神经网络结构和与之配套的误差传播/激励传播算法可能会从另一个方向优化神经网络的训练。这种优化可能达到量化感知
技术无法实现的程度。

综观工程技术发展历史,最底层的算法和技术迭代往往艰深险难,每进步一毫可能都要耗费巨大的时间人力成本,更甚者算法层面
的进步可能需要数学理论的艰难创新或是天才般的灵光乍现。本文集成了快速傅立叶变换算法、低功耗约束的高效规划算法、量化感知
加速技术等诸多经典或者前沿的工程算法,其中不妨极具影响力的`世界级'算法。经过软硬件整合实现一套可以在华为系硬件芯片上
运行的完整的自适应低功耗有限长单位脉冲响应滤波算法(Self-Adjusting Low-Dissipation Finite Impulse Response Filter)

计划中附录中包含本项目工程的源代码实现以及注释;量化感知技术实现的神经网络框架(基于PyTorch/Tensorflow)源代码实现
和接口注释;以及详细的操作方式。(API Source Code;Description;Manual)

如果有后续的经费支持,我们团队认为我们可能继续维护该项目的开发,并且生产出更加高效的算法。

特别地,我们团队将集中处理利用神经网络进行特定场景的自适应滤波器算法设计的研究和实现上。
因为在我们近一个月的调研和测试中,我们发现底层算法的性能突破很大程度依赖于硬件自身,另一大部分是要求设计者对
离散调和分析理论有极深的造诣。这才有可能设计出超过本文第二三部分所述的经典或部分改良的底层理论算法。

未来本团队可能的进一步研究方向包括如下几个方面:
\begin{enumerate}
    \item 完善 FIR 滤波算法的不同场景的适应性,以及去耦合、模块化整个 ACFIR 滤波程序项目。
    \item 接口专家提供数据集、构建神经网络对 FIR 滤波器自适应功能进行训练。
    \item 更完善的前端呈现、包括基于 Qt 的网络训练可视化和接口化的 FIR Filter 硬件架构/测试可视化工程。
\end{enumerate}

后续的难题攻克需要更多的技术支持和经费支持,也希望本文回答了此题目的技术诉求。期待能够得到华为技术有限公司的后续资助。

本团队的全部项目代码均按照 MIT License 开源。

项目仓库地址是 
\href{https://github.com/YOTALTEAM/HUAWEI_FIR}{https://github.com/YOTALTEAM/HUAWEI\_FIR}。
欢迎有关人士对项目提出宝贵意见建议。