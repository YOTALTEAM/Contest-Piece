\section{\heiti{总结及展望}}


有限长单位脉冲响应滤波算法是一种广泛应用在通信工程、音视频处理、信号处理等工程方向的硬件算法。其作为信号处理技术中
的核心主流算法,已经被多系列主流基带芯片通信芯片所集成。如何高效执行有限长单位脉冲响应滤波算法,无论是从算法理论上
还是硬件兼容软件集成上都是学界和工业界的主攻难题。

量化感知是近二三十年来发展迅速的加速神经网络训练的主流技术之一。从提出至今已经有不同研究者提出了种类繁多的量化策略
其想要解决的主要问题是在一些性能限制下最小化如下的量化后数据和原始数据的偏差

\begin{equation*}
    \min \sum_{r} \| Q(r) - r\|_2,\qquad \mathrm{s.t.}\quad \mathrm{Constraints(r) = 0}
\end{equation*}

高效的量化感知算法可以大大加快神经网络训练的底层运算速度,是一个加速神经网络训练的可行方向。但是另一方面,
神经网络自身的结构优劣和误差传播算法的设计好坏也是制约其训练质量和速度的重要指标。从这一意义上说,设计更好的
神经网络结构和与之配套的误差传播/激励传播算法可能会从另一个方向优化神经网络的训练。这种优化可能达到量化感知
技术无法实现的程度。

综观工程技术发展历史,最底层的算法和技术迭代往往艰深险难,每进步一毫可能都要耗费巨大的时间人力成本,更甚者算法层面
的进步可能需要数学理论的艰难创新或是天才般的灵光乍现。本文集成了快速傅立叶变换算法、低功耗约束的高效规划算法、量化感知
加速技术等诸多经典或者前沿的工程算法,其中不妨极具影响力的`世界级'算法。经过软硬件整合实现一套可以在华为系硬件芯片上
运行的完整的自适应低功耗有限长单位脉冲响应滤波算法(Self-Adjusting Low-Dissipation Finite Impulse Response Filter)

附录中包含了可执行的C++工程的源代码实现以及详细注释;量化感知技术实现的神经网络框架(基于OpenNN/Tensorflow)源代码实现
和接口注释;以及详细的操作方式。(API Source Code;Description;Manual)

如果有后续的经费支持,我们团队认为我们可能继续维护该项目的开发,并且生产出更加高效的算法。