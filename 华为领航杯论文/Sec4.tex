\section{\heiti{实验数据及软硬件介绍}}

为了进行公平的比较,我们在 Xilinx Artix-7 FPGA平台上,
分别对 $64$、$128$ 和 $256$ 点的 FFT 硬件设计进行了实验,以探索速度和面积效率之间的不同权衡。
为了考虑 DSP 和 BRAM,我们使用 Slice 等效成本(SEC)作为面积评估的度量。
通过 $\mathrm{SEC=\# BRAM x 200+\# DSPsx 100+\# Slice}$ 来计算 SEC,即一个 DSP 块和一个 36
 Kb BRAM 分别等于 $102.4$ 和 $196.2$ 片。
面积效率指标则由面积时间积(ATP)评估,
计算为\# SEC $\times$ FFT 算法执行时间。
我们将所提出的高效 FFT 流处理器性能与近期内和``FIR''、``多项式乘法''、
``傅里叶变换''和``蝶形运算''等主题相关的硬件工作做对比。
受益于我们所采用的冗余计算方案,
即使在高并行度的执行环境下,
我们的设计仍然可以跑到 $200$ Mhz、$177$ Mhz和 $162$ MHz。
为与所引工作进行公平比较,我们针对 $N=256$ 点的 FFT 做了实现,
其 ATP 评估结果为 $1813.7$,位列相关工作最优。
此外,为充分论证设计在不同并行度下的执行性能和面积、速度权衡,我们也针对 $N=64$ 以及 $128$ 做了实现,以供参考。
基于实现结果,我们推荐使用 $N=32\sim 128$ 的参数来加速 FIR 应用问题,以取得最佳的加速效果。