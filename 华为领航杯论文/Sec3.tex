\section{理论及算法}

\subsection{离散傅立叶变换(Discrete Fourier Transformatioin, FFT) 的快速实现}

快速傅立叶变换(通常只是至离散情形的)
以下内容参考wikipedia 。如



第一种办法:Wingoard 加速算法(Wingoard Fourier Transformation,WFT)

第二种办法:Cooley-Tukey 算法(Cooley-Tukey Algorithms)

此方式适用于 $n= 2^m$ 时,这里的实现方式主要参考了算法导论一书的 FFT 章节见 \cite[Polynomials and FFT]{Cormen2022}

另外更加高效的办法是参考 \cite{Li2010}。或者关于快速傅立叶变换的综述性质书籍\cite{Nussbaumer1982}。

第三种办法
\subsection{乘法器加速的 Booth 算法(Booth Multiplication Algorithms)}


\subsection{量化感知中的混合精度量化(Mixed Precision Quantization)}


