\section{\heiti{理论及算法}}

\subsection{\heiti{离散傅立叶变换(Discrete Fourier Transformatioin, FFT) 的快速实现}}

快速傅立叶变换(通常只是至离散情形的)
以下内容参考wikipedia 。如



第一种办法:Wingoard 加速算法(Wingoard Fourier Transformation,WFT)

第二种办法:Cooley-Tukey 算法(Cooley-Tukey Algorithms)

此方式适用于 $n= 2^m$ 时,这里的实现方式主要参考了算法导论一书的 FFT 章节见 \cite[Polynomials and FFT]{Cormen2022}

另外更加高效的办法是参考 \cite{Li2010}。或者关于快速傅立叶变换的综述性质书籍\cite{Nussbaumer1982}。

第三种办法

\subsection{\heiti{乘法器加速的 Booth 算法(Booth Multiplication Algorithms)}}

Karatsuba 算法

以下对于高精度大整数乘法的 Karatsuba 算法简介摘录自 \href{https://oi-wiki.org/math/bignum/#karatsuba-%E4%B9%98%E6%B3%95}{OI Wiki:Karatsuba Algorithm}。

如果取高精度数字(必要时同时约化为大整数)的位数为 $n$,那么高精度—高精度竖式乘法需要花费 $\O(n^2)$ 的时间。
本节介绍一个时间复杂度更为优秀的算法,由前苏联(俄罗斯)数学家 Anatoly Karatsuba 提出,是一种分治算法。

考虑两个十进制大整数 $x$ 和 $y$,均包含 $n$ 个数码并可以有前导零。任取 $0 < m < n$,记
\begin{eqnarray*}
    \begin{aligned}
        x &= x_1 \cdot 10^m + x_0, \\
        y &= y_1 \cdot 10^m + y_0, \\
        x \cdot y &= z_2 \cdot 10^{2m} + z_1 \cdot 10^m + z_0,
    \end{aligned}
\end{eqnarray*}
其中 $x_0,y_0,z_0,z_1 < 10^m$。简单四则运算可得
\begin{eqnarray*}
    \begin{aligned}
        z_2 &= x_1 \cdot y_1, \\
        z_1 &= x_1 \cdot y_0 + x_0 \cdot y_1, \\
        z_0 &= x_0 \cdot y_0.
    \end{aligned}    
\end{eqnarray*}
继续观察(本质上使用分配律或 Wingoard 加速算法)将
\begin{eqnarray*}
    z_1 = (x_1 + x_0)(y_1 + y_0) - z_2 - z_0
\end{eqnarray*}
从而只要计算出 $(x_1+x_0), (y_1 + y_0)$ 再与 $z_2,z_0$ 相减即可求出 $z_1$

因为普遍讲计算机上乘法的耗时要大于甚至远大于加法(尤其是数位越大的时候)。从而将乘法替换为
加法会大大降低算法的渐进时间复杂度。

上式实际上是 Karatsuba 算法的核心,
它将长度为 $n$ 
的乘法问题转化为了 3 个长度更小的子问题。
若令 
\begin{eqnarray*}
    m = \left\lceil \dfrac n 2 \right\rceil
\end{eqnarray*} 
记 Karatsuba 算法计算两个 $n$ 位整数乘法的耗时
为 $T(n)$,则有 
\begin{eqnarray*}
    T(n) = 3 \cdot T \left(\left\lceil \dfrac n 2 \right\rceil\right) + \O(n)
\end{eqnarray*}
由主定理可得 
\begin{eqnarray*}
    T(n) = \O (n^{\log_2 3}) \approx \O (n^{1.585})
\end{eqnarray*}
这相比于通常的大整数乘法的 $\O (n^2)$ 有数量级的减少。也回答了 20世纪 60年代 Kolmogorov 关于大整数乘法渐进时间复杂度是否
为 $\Omega(n^2) \approx \O (n^2)$ 的重大问题。

而 Anatoly Karatsuba 发明这个算法时年仅 23 岁,仅仅是在 Kolmogorov 陈述他关于任何大整数乘法算法的
时间复杂度趋于 $\O(n^2)$ 这一猜测一周后 Karatsuba 就完成了上述更快算法的构造。具体的历史信息可参考 
\href{https://en.wikipedia.org/wiki/Karatsuba_algorithm}{Wikipedia:Karatsuba Algorithm}。其中一段摘录如下

\begin{quote}
    In 1960, 
    Kolmogorov organized a seminar on mathematical problems in cybernetics 
    at the Moscow State University, where he stated the $\Omega (n^2)$ conjecture and other problems 
    in the complexity of computation. Within a week, Karatsuba, then a 23-year-old student, found an
     algorithm that multiplies two n-digit numbers in $\O (n^{\log_2 3})$ elementary steps, thus 
    disproving the conjecture. Kolmogorov was very excited about the discovery; 
    he communicated it at the next meeting of the seminar, which was then terminated. 
    Kolmogorov gave some lectures on the Karatsuba result at conferences all over the world 
    (see, for example, ``Proceedings of the International Congress of Mathematicians 1962'', 
    pp. 351–356, and also ``6 Lectures delivered at the International Congress of Mathematicians 
    in Stockholm, 1962'') and published the method in 1962, in the Proceedings of the USSR Academy 
    of Sciences. The article had been written by Kolmogorov and contained two results on 
    multiplication, Karatsuba's algorithm and a separate result by Yuri Ofman; it listed 
    \textquotedbl A. Karatsuba and Yu. Ofman\textquotedbl as the authors. Karatsuba only became aware of the paper 
    when he received the reprints from the publisher.
\end{quote}

更完整的第一手算法理论参考可见 Karatsuba 本人的综述 \cite{Karatsuba1995}。本项目使用其思想实现了针对滤波器乘法的版本如下,
主要集成了 Karatsuba 算法实现了多项式乘法和卷积,最后再处理所有的进位问题。

\subsection{\heiti{量化感知中的混合精度量化(Mixed Precision Quantization)}}

此部分主要参考 \citetitle{Gholami2021} 的大量参考文献中实现的不同的量化感知技术。其中

在神经

在这篇文章中