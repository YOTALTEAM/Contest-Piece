\section{问题背景阐述}

\subsection{有限脉脉冲响应滤波器(Finite Impulse Response Filter, FIR Filter)}

工程上进行
FIR 滤波器(FIR Filter,以下简称FIR或FIR滤波器)是非递归型滤波器的简称,全称是有限长单位脉冲响应滤波器
(Finite Response Filter)。

其中带有常数系数的
如下图




\subsection{离散傅立叶变换及快速实现(Discrete/Fast Fourier Transformatioin, DFT/FFT)}

如果采用上述一般形式的离散 Fourier 变换公式在硬件上进行运算,那么完成一次完整的 DFT 需要进行
$\O (n^2)$ 次乘法和 $\O(n^2)$ 次加法。其中 $n$ 代表 DFT 变换器的阶数(或者滤波器的抽头数)

在20世纪60年代,来自 Princeton 的计算机科学家 Cooley 和数学家 Tukey 发明了一种基于快速幂算法和 Gauss 和算术性质
的快速离散 Fourier 算法。其可以通过数量级地减少 DFT 算法中的乘法次数
大大加快原有离散 Fourier 变换的运算速度。本质上这一算法的理论框架已经被德国数学家 Carl Friedrich Gauss 于 19世纪初
发现并证明,参见。

\subsection{滤波器算法中的乘法器性能}


\subsection{技术诉求}


