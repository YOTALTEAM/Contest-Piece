\section{\heiti{问题背景阐述}}

\subsection{\heiti{有限脉脉冲响应滤波器(Finite Impulse Response Filter, FIR Filter)}}

工程上进行
FIR 滤波器(FIR Filter,以下简称FIR或FIR滤波器)是非递归型滤波器的简称,全称是有限长单位脉冲响应滤波器
(Finite Response Filter)。

其中带有常数系数(或自适应系数)的数据流程图如下

\paragraph{FIR 滤波器的架构优化}

\paragraph{FIR 滤波器中的内存管理}


对于更加详细的信号处理介绍(尤其是)

\subsection{\heiti{离散傅立叶变换及快速实现(Discrete/Fast Fourier Transformatioin, DFT/FFT)}}

滤波器实现了经典的 Cooley-Tukey 构造的快速傅立叶变换算法(Fast Fourier Transformatioin, FFT)。实现细节参考附录
或\href{https://en.wikipedia.org/wiki/Cooley–Tukey_FFT_algorithm}{Wikipedia}。

如果采用上述一般形式的离散 Fourier 变换公式在硬件上进行运算,那么完成一次完整的 DFT 需要进行
$\O (n^2)$ 次乘法和 $\O(n^2)$ 次加法。其中 $n$ 代表 DFT 变换器的阶数(或者滤波器的抽头数)

在20世纪60年代,来自 Princeton 的计算机科学家 Cooley 和数学家 Tukey 发明了一种基于快速幂算法和 Gauss 和算术性质
的快速离散 Fourier 算法。其可以通过数量级地减少 DFT 算法中的乘法次数
大大加快原有离散 Fourier 变换的运算速度。本质上这一算法的理论框架已经被德国数学家 Carl Friedrich Gauss 于 19世纪初
发现并证明,参见。

\subsection{\heiti{滤波器算法中的乘法器性能}}

本项目同时采用了冗余计算理论(主要是近似位宽乘法)和 tooth 快乘提高 FIR Filter 中的乘法器性能。以下是简要说明

\subsection{\heiti{技术诉求}}

针对华为领航杯应用数学大赛的难题``低功耗自适应 FIR 滤波器''有如下三个主要技术诉求。

\begin{enumerate}
    \item 诉求一:通过改变滤波器架构设计等办法,使得其硬件算法整体性能提升40\%。
    \item 诉求二:指定位宽下引入近似乘法,实现乘法资源降低50\%,误差或信噪比(SNR)大于 $75 \mathrm{dBc}$
            ;或者使用机器学习模型训练,使得其训练
            性能基本无损($< 0.1 \mathrm{dB}$)
    \item 诉求三:" 低位宽 " FIR 滤波器算法实现($< 6 \mathrm{bit}$),并能够实现模型稳定校正且性能基本无损($< 0.1 \mathrm{dB}$)
\end{enumerate}

\paragraph{诉求一}

就本团队理解。衡量 FIR 滤波器的整体性能指标可以使用如下的面积来确定。

\paragraph{诉求二}

利用 Shannon 信息熵计算公式,本项目通过基准测试方案在仿真机上对 FIR 滤波算法进行了...的测试

通过比较。。。


\paragraph{诉求三}

通过理论计算。一般而言参考刘伟强等人的综述可知信噪比计算公式如下

\begin{eqnarray*}
    \mathrm{SNR} := 10 \log_{10} A/B
\end{eqnarray*}

经过如下的理论计算可知实现位宽的理论值应该为 $12 \mathrm{bit}$。
通过更大量的 ``代价权衡'' 将滤波器的位宽降低到 $6 \mathrm{bit}$,本项目采用自适应机器学习预测的办法。此方式将会在论文第三部分
详细论证


在本文第二部分经典方法复现中,三个子节将会依次详细论述本项目对上述三个诉求的经典解决方式。

在第三部分``理论和算法''中本文将尽可能清晰地叙述项目中使用的关键算法理论,具体创新和优化的细节部分。

在第四部分``实验数据及软硬件介绍''和第五部分``实验结果及主要结论''中本文将展示根据前三部分的算法理论实现的工程在仿真环境
的测试数据以及指标比较。

最后关于项目的复现以及移植,完整的源代码解释和运行环境信息全面呈现在附录一节。

