\section{\heiti{经典方法复现}}

\subsection{\heiti{基于 Cooley-Tukey FFT 算法实现的 FIR 滤波器}}

具体理论可参考维基百科。

\subsection{\heiti{冗余计算和 Booth 乘法}}

Booth 乘法器的原生算法可以在保持设定精度的前提下显著地提升滤波器中乘法器的性能。这里简要介绍 Booth 乘法的实现方法。


采用静态优化技术设计合适的滤波器

设计合适的乘法位宽


分布合适数量的乘法器

\subsection{\heiti{基于量化感知的自适应加速}}

针对于
由于在全精度滤波器中乘法器进行的是浮点运算

小规模硬件(如基带芯片)上布局神经网络要求低缓存和低延迟。
这样的算法才能满足硬件上信号处理的速度和硬件设备自身的条件限制。

量化感知技术(Quantization Perception)可以减轻神经网络训练上浮点数计算的时间/空间复杂度,从而释放算力进而加速
神经网络的训练。

对于浅层神经网络而言采用前向直接估计器 (STE)

而使用非前向直接估计器。

有关于自适应滤波器 FIR 滤波器
